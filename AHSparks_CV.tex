%!TEX TS-program = xelatex
\documentclass[]{friggeri-cv}

\addbibresource{bibliography.bib}

\begin{document}

\header{AdamH.}{Sparks}
       {Plant Disease Management Specialist@IRRI}


% In the aside, each new line forces a line break
\begin{aside}
\section{skills}
\small{GIS
modelling
agricultural statistics
~
\section{contact}
\href{http://www.irri.org/}{IRRI}
Los Ba\~nos, Laguna
Philippines
~
\emph{Mail:}
DAPO Box 7777
Metro Manila
1301 Philippines
~
\href{mailto:a.sparks@irri.org}{a.sparks@irri.org}
~
+63 (2) 580 5600 \faPhone
+63 908 182 8012  \faMobilePhone
{\href{skype:adam.h.sparks?call}{adam.h.sparks \color{skype}{\FA \symbol{"F17E}}}}
~
\section{web}
\href{http://plus.google.com/+AdamHSparksPhD}{+AdamHSparksPhD {\color{google.plus}\faGooglePlusSign}}
\href{https://www.twitter.com/adamhsparks/}{@adamhsparks {\color{twitter.blue}\faTwitter}}
\href{https://github.com/adamhsparks/}{adamhsparks \faGithub}}
\end{aside}

\section{personal summary}
I am a plant pathology epidemiologist and ecologist who can effectively communicate and collaborate with diverse partners and stakeholders in an international setting. My work encompases traditional field-based research, epidemiological modeling, climate change, GIS and statistical methods. In turn, I use this work to understand what drives crop disease epidemics and derive disease control recomendations farmers and make recomendations for policy makers and other decision makers.

\section{experience}
\begin{entrylist}
  \entry
    {since 2012}
    {International Rice Research Institute \normalfont{(IRRI)}}
    {Scientist I}
    {\emph{Develop tools and strategies for farmers to use in addressing rice diseases}}
  \entry
    {2011-2012}
    {International Rice Research Institute \normalfont{(IRRI)}}
    {Post-Doctoral Fellow}
    {\emph{Linked plant disease models with GIS tools}}
  \entry
    {2009-2010}
    {Kansas State University}
    {Post-Doctoral Research Associate}
    {\emph{Developed and refined predictive Fusarium head blight models for wheat}}
  \entry
   {2002-2004}
   {University of Nebraska-Lincoln}
   {Research Technologist}
   {\emph{Managed maize and soybean plant pathology extension field research}}
  \entry
   {2000-2003}
   {University of Nebraska-Lincoln}
   {Research Technician}
   {\emph{Managed maize and sorghum plant pathology extension field research}}
  \entry
   {1999-2000}
   {Purdue University}
   {Assistant Director}
   {\emph{Coordinated training events for Purdue Diagnostic Training and Research Center}}
  \entry
   {1997-1999}
   {Purdue University}
   {Research Technician}
   {\emph{Managed soybean and canola production research studies}}
\end{entrylist}

\section{education}

\begin{entrylist}
  \entry
    {2009}
    {Ph.D. {\normalfont Plant Pathology}}
    {Kansas State University, Manhattan, KS}
    {Plant Disease Epidemiology and Ecology}
  \entry
    {}
    {Dissertation: \normalfont{\emph{Disease risk mapping with metamodels for coarse resolution}}}
    {}
    {\emph{predictors: global potato late blight risk now and under future climate conditions}}
 \entry
    {2007}
    {Graduate Certificate {\normalfont Geography}}
    {Kansas State University, Manhattan, KS}
    {Geographic Information Science}
  \entry
    {2000}
    {B.Sc. {\normalfont Agronomy}}
    {Purdue University, West Lafayette, IN}
    {Soil and Crop Management}
\end{entrylist}


\section{publications}

\printbibsection{article}{peer-reviewed}
\begin{refsection}
  \nocite{*}
  \printbibliography[sorting=chronological, type=inreview, title={in review}, heading=subbibliography]
\end{refsection}
\begin{refsection}
  \nocite{*}
  \printbibliography[sorting=chronological, type=inproceedings, title={conferences/proceedings},  heading=subbibliography]
\end{refsection}
\begin{refsection}
  \nocite{*}
  \printbibliography[sorting=chronological, type=inbook, title={book chapters}, heading=subbibliography]
\end{refsection}
\begin{refsection}
  \nocite{*}
  \printbibliography[sorting=chronological, type=report, title={reports}, heading=subbibliography]
\end{refsection}

\section{invited talks}
\begin{entrylist}
\entry
{2013}
{Biosecurity Risks in Southeast Asia Impacting on Human Food Supplies}
{}
{Forum: Pacific Environmental Safety Forum Australian Department of Defence and U. S. Pacific Command\\Sydney, New South Wales, Australia}

\entry
{2010} 
{Global potato late blight risk in response to climate change, possible futures for a historic disease}
{}
{Symposium: Emerging Infectious Diseases in Response to Climate Change.\\New York Academy of Sciences, New York, New York}
\end{entrylist}

\section{extramural support}
  \begin{entrylist}
  \entry
	{2013-2017}
	{PRISM \normalfont{Philippine Rice Information SysteM}}
	{\$2,765,783}
	{Component B -- Crop Health Monitoring, Co-PI: A Nelson}
  \entry
	{2013-2015}
	{Syngenta}
	{\$454,640}
	{Phase II, Project 2 -- Crop Health Management}
  \end{entrylist}


\section{professional certifications}
PRINCE2 Foundation (2014) candidate number: P2R/009385 – HiLogic  Pty Ltd.

\section{professional affiliations}
American Phytopathological Society (APS)

Australasian Plant Pathology Society (APPS) 

International Society of Plant Pathology (ISPP)

International Association for the Plant Protection Sciences (IAPPS)

\end{document}