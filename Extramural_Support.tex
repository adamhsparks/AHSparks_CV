\section*{Extramural Support}
  \begin{entrylist}
      \entry{2019-2020}
            {FAO Consultancy: Technical Consultant for Developing an In-country Agrometeorolgy Bulletin for Cambodia}
            {AUD 80,000}
            {PI: A. Sparks}
      \entry{2019-2021}
			{GRDC Post-doctoral Fellowship Project USQ1903: A model for predicting chickpea ascochyta blight risk, Parent Project: DAW1810}
			{AUD 510,800}
			{PI: A. Sparks}
	  \entry{2018-2021}
			{GRDC Research Project DAW1810: Disease epidemiology and management tools for Australian grain growers}
			{AUD 650,429}
			{Subcontract from DPIRD WA to USQ PI: A. Sparks}
	  \entry{2018-2021}
			{RIRDC Project PRJ-010814: Northern rice Australia - Developing rice growing packages for tropical climates}
			{AUD 2,074,773}
			{PIs: G. Ash, K. Pembleton, A. Sparks}
	  \entry{2017}
			{GRDC Research Project DAQ00186: Improving grower surveillance, management epidemiology knowledge and tools to manage crop disease\\1-Year Extension to Existing Project}
			{AUD 1,237,992}
			{PI: A. Sparks}
      \entry{2017}
            {USQ Research Infrastructure Program 2017}
            {AUD 25,000}
            {PIs: D. Adorada, A. Sparks, A. Young}
      \entry{2016-2019}
            {EPIC \normalfont{(Developing Ecologically-based Participatory IPM package for rice in Cambodia)}}
            {USD 2.2 million}
            {PIs: B. Hadi (IRRI), A. Sparks, V. Kumar (IRRI), A. Stuart (IRRI), R. Oliva (IRRI), I.R. Choi (IRRI)}
      \entry{2016--2019}
	        {Syngenta-IRRI Scientific Knowledge and Exchange Program}
	        {USD 484,274}
	        {Phase III, Sub-Project 1 - Crop Health Management\\PI: A. Sparks (IRRI) and K. K. Fui (Syngenta)}
      \entry{2015--2017}
            {Identifying resistant rice germplasm to false smut using combined screening approaches and understanding the mechanisms underlying rice resistance}
            {USD 653,914}
            {Epidemiology and environmental characterisation of false smut,\\PIs: B. Zhou (IRRI), C. M. {Vera Cruz} (IRRI) and A. Sparks (IRRI)}
      \entry{2013--2017}
            {Philippine Rice Information SysteM (PRISM)}
	        {USD 2.8 million}
	        {Component B - Crop Health Monitoring,\\PIs: A. Nelson (IRRI), A. Sparks (IRRI), G. S. Arida (PhilRice), E. J. P. Quilang (PhilRice)}
      \entry{2013--2015}
	        {Syngenta-IRRI Scientific Knowledge and Exchange Program}
	        {USD 454,640}
	        {Phase II, Sub-Project 2 - Crop Health Management\\PI: A. Sparks (IRRI) and K. K. Fui (Syngenta)}
  \end{entrylist}

\section*{Awards}
  \begin{entrylist}
	\entry{2019}
		{Theo Murphy (Australia) Initiative for support for ‘Re:produce – kick-off meeting of reproducible research network’}
		{AUD 17,000}
		{With R. Panczak (P.I.), P. Baker, F. Gacenga, R. King, L. Li, J. Lodge, C. Lim and N. Schnyder}
  	\entry{2016}
    	{GovHack 2016 First Place Award for Paddock to Plate Category, John Conner Hack}
    	{}
    	{As part of the Toowoomba Trio with K. Pembleton and G. Grundy}
  \end{entrylist}
